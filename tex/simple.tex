\documentclass{article}

\setlength{\parindent}{1,25cm}

\author{Сыров Александр Викторович}
\usepackage[utf8x]{inputenc}  
\usepackage[russian]{babel}  
\usepackage{graphicx}
\usepackage{float}
\usepackage{wrapfig}
\usepackage{indentfirst}
\usepackage{amsmath}
\begin{document}

\section{Инструкция по развертке локальной сети}

Автор данной PDF: Сыров Александр Викторович.

\subsection{Введение}

Локальная сеть на данный момент является важной составляющей любого предприятия, использующего в своей работе компьютеры. Локальная сеть позволяет упростить многие бизнес-процессы, повысить контроль за качеством работы.

Целью данной работы является написание инструкции по развертыванию локальной сети в типографии. Необходимо исследовать предприятие, на котором будет развертываться локальная сеть, его бизнес-процессы и потоки данных, определить, какое необходимо оборудование и программное обеспечения для успешного развертывания локальной сети.

\subsection{Общее описание}

Типография, для которой требуется развернуть локальная сеть, занимается производством различной типографской продукции. Локальная сеть необходима для организации полного цикла рабочего процесса от заказа клиента до готового изделия.

Данное предприятие располагается в двух зданиях: офисе и цехе. Офис состоит из 3 кабинетов, предназначенных для части работников типографии. Цех находится на определенном удалении от офиса, более 1 км. Цех состоит из 4 больших комнат с различными станками.

\newpage

\begin{figure}[h!]
	\includegraphics[scale=0.3]{../img/tree.png}
	\caption{Квадратная елка}
\end{figure}

\begin{equation} \label{eu_eqn}
    e^{\pi i} + 1 = 0
\end{equation}

Замечательное равенство \ref{eu_eqn} известно также как тождество Эйлера.

\newpage

\tableofcontents

\end{document}